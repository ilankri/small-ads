\documentclass[12pt,english,french]{article}
\usepackage[utf8]{inputenc}
\usepackage[T1]{fontenc}
\usepackage{lmodern}

\usepackage{babel}
\babeltags{eng=english}
\newcommand{\texten}[1]{\texteng{\emph{#1}}}
\frenchbsetup{ItemLabels=\textendash{}, og=«, fg=»}
\usepackage{babelbib}

\usepackage[margin=2cm]{geometry}
\usepackage[final,babel]{microtype}
\usepackage[np,autolanguage]{numprint}
\usepackage[np]{nbaseprt}

\usepackage{mathtools}
\usepackage{amssymb}
\usepackage[hidelinks]{hyperref}

\date{\today{}}
\author{Belaid Lagha \and{} Idir Lankri}
\title{Projet Protocoles des services Internet}

\begin{document}
\maketitle{}

\section{Architecture du serveur}
Le serveur attend les connexions des clients sur le port $1027$ comme
spécifié dans le protocole.  Le serveur accepte les connexions jusqu'à
ce que le nombre de limite de connexions ouvertes soit atteint.  Par
défaut, on a fixé arbitrairement cette limite à $100$ mais on a la
possibilité de faire varier ce nombre en lançant le serveur avec un
argument en ligne de commande (voir le fichier \texttt{README}).

Une fois la connexion entrante acceptée, le serveur lance un
\texten{thread} (voir la classe \texttt{server.ServerThread}) qui va traiter
les requêtes envoyés par ce client.  Le serveur peut donc (et
heureusement) gérer plusieurs clients à la fois.

Pour stocker les annonces sur le serveur, on utilise une base de données
\texten{in-memory} (classe singleton \texttt{server.DB}).  Cette base de
données vise simplement à maintenir une association entre une adresse IP
et un utilisateur (modélisé par la classe \texttt{common.User}).  Cette
base de données pouvant être utilisée de manière concurrente par les
différents \texten{threads} du serveur, on a pris soin d'en protéger les
accès.

\section{Architecture du client}
Le programme client est composé d'une boucle interactive principale qui
exécutent les commandes du client : se \texten{logger}, lister les
annonces, poster une annonce, supprimer une annonce, envoyer un message
à un autre client, lister les messages reçus et se déconnecter.

La gestion des messages échangés entre client est faite de manière
similaire à une boîte mail.  La réception des messages provenant des
autres clients est faite de manière asynchrone, c'est-à-dire qu'on a un
\texten{thread} en arrière plan qui s'occupe de gérer la réception des
messages UDP des clients (voir la classe \texttt{Mailbox.Inbox}).

\section{Analyse de sécurité}
Le serveur est \emph{a priori} toujours disponible\dots{} tant qu'il
a suffisamment de mémoire puisqu'on utilise une base de données en
mémoire.

\end{document}

%%% Local Variables:
%%% mode: latex
%%% TeX-master: t
%%% End:
